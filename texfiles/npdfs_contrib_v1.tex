%\documentclass[prb,11pt]{revtex4}
%\documentclass[twocolumn]{revtex4}

\section{Nuclear PDFs\label{sec:nPDFs}}
%
Nuclear parton distributions functions (nPDFs) are the first step in understanding the behavior of nuclear matter at the elementary particle level. Moreover they play a crucial role in the determination of the free proton PDFs, as nuclear targets are routinely used for separating the different partonic flavors in PDFs fits. Notwithstanding their importance, nPDFs are not as well known as the free proton case, primarily due to two factors. First, despite the phenomenal success of HERA in determining the proton structure, no electron-nucleus collider has ever existed. Second, only a few nuclei have been studied in detail and the data span a limited region of the kinematic space, to the point that the only constrained nuclear distributions are the distributions quarks in the mid-$x$ region.  For regions where data are not readily available, extrapolations for $x < 10^{-3}$ employ the fulfillment of the charge and momentum sum rules and, at mid to high-$x$, the sea quarks and gluon densities are often determined by ad-hoc assumptions during the fitting procedure, rather than from actual constraints from the data.  The parametrizations of the nuclear effects in the neutron frequently employ the assumption of isospin symmetry, which is difficult to validate given the nearly isoscalar nature of most available nuclei.

Less than about one third of the data used in nPDF fits come from heavy nuclei which complicates the possibility of truly separating the nuclear modifications for each flavor. A possible path for flavor separation would be using charged current (CC) data from DIS, which is available for iron and lead, where the cross-section depends on different combinations of the PDFs than the neutral current (NC) processes. It has been also suggested that nuclear effects might not be universal and therefore making a truly global fit of the nPDFs would not be possible. Up to now and within experimental uncertainties, NLO fits including NC and CC data have not shown visible tension. Unfortunately these fixed target experiments cover a very limited region of the kinematic space and are lacking in precision, not allowing yet for a conclusive answer.


The unexplored low-$x$ region, dominated by the gluon density, opens the possibility of finding new non-linear phenomena such saturation, and puts to test the applicability range of the linear regime. The other extreme of the kinematic space, the high-$x$ region, is of particular interest as there appears the first measured sign of nuclear effects in high energy collisions: the EMC effect. Moreover, for beyond the Standard Model searches at the LHC, rare high-$x$ gluon initiated events could be enhanced. However the nuclear gluon is difficult to access at high-$x$, and great care has to be put in its determination. Despite the lack of data, the determination of nPDFs in all the kinematic space is of crucial relevance and thus the target of several efforts. Given the fact that sea quarks and gluon densities are tied through the DGLAP evolution equations, studying processes sensitive to either sea quarks or gluons has an impact on our knowledge of nPDFs.    

\subsection{Accessing the Nuclear Gluon at High-$x$}

One of the observables in which initial state gluons can account for most of the cross-section are jets. Unfortunately the LHC measurements in jets in $p+\mathrm{Pb}$ collisions by the ATLAS collaboration~\cite{ATLAS:2014cpa} have been integrated over in the $0-90\%$ centrality bin instead of as the customary minimum bias data, rendering the quantity difficult to compare to in the context of collinear factorized pQCD. However, the data of the di-jets measured by the CMS collaboration were published~\cite{Chatrchyan:2014hqa} and further included in the latest NLO nPDF analysis of EPPS16~\cite{Eskola:2016oht}. There it was shown that, while at $Q^{2}$ where some information is lost in the PDF evolution, the di-jets from CMS have a non-negligible impact on the high-$x$ gluon distribution. As the EPPS16 fit comprises about $2000$ data points and allows for more flexible parametrizations, the effect of the di-jet data on constraining the gluon is less effective. %\footnote{Raphael: Is this true or just because there is little data constraining gluon distributions? I think the dilution argument is a bit spurious.}.   SPR:  I took this to just mean it has less impact. 

Nonetheless, jets remain a relevant tool to access the gluon density. In recent works~\cite{PhysRevD.95.094013, PhysRevD.97.114013} it has been shown that (di-~)jets in $e+A$ collisions at a future Electron-Ion Collider (EIC) have the potential to reduce the theoretical uncertainties by an order of magnitude at both low and high-$x$, reaching into the anti-shadowing and EMC effect regions.

A complementary way of accessing the high-$x$ gluon is using the charm quark structure function. This quantity is determined by tagging the charm in the final state and theoretically has its leading order perturbative contribution from the photon-gluon fusion process. In addition, this observable might hold the key to study if there is an intrinsic content of charm in the proton or nucleus or if heavy quarks appear only by radiation from the gluons, though fully disentangling the gluon and intrinsic charm is challenging. The studies of DIS reduced cross-section with simulated EIC data and its impact on the gluon nPDF~\cite{PhysRevD.96.114005} show that the inclusive data could reduce the uncertainty bands up to a factor of~4 at low $x$, while the charm reduced cross-section would have a dramatic effect, diminishing the uncertainties by almost an order of magnitude at high-$x$.   

\subsection{Improvements for nPDFs}

Future programs will play a key role on determining the nPDFs and it is crucial to incorporate lessons learned from prior experiments, publications, and analysis. A simple example is results from HERA where initial data were published in the form of the structure functions $F_{2}$, while in later years and after the conclusion of data taking, new publications were made of the cross-sections. The shift came from the realization that the structure functions are not a fundamental observable if considering the gluon densities and potential saturation effects. This adds uncertainty in $F_{2}$ and no reanalysis under this realization has been performed.  Similarly, assumptions are made about the form of the longitudinal structure functions $F_\mathrm{L}$ and become deeply ingrained in the results.  Corrections 
to account for the non-isoscalarity of some targets make assumptions on the flavor-dependence of the modification in nuclear data analyses and can lead to very different shapes for the EMC effect.  In this light it would be extremely beneficial for the community to publish future results which include the measured cross sections, as well as the extracted structure functions, with and without corrections or assumptions. 
