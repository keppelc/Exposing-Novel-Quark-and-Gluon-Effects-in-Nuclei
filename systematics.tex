\section{Systematics}
\subsection{Nuclear Dependence of $R=\sigma_L/\sigma_T$}


Due to the relatively low energy of the 6~GeV JLab E03-103 measurements, effects due to acceleration and
deceleration of electrons in the Coulomb field of the heavier targets (Cu and Au) could not be ignored.
However, once applied, these so-called Coulomb corrections resulted in ratios systematically larger
than those found by previous experiments.  This apparent discrepancy motivated the re-examination of
earlier measurements, and it was found that, while the bulk of the large $x$ measurements from SLAC
were taken at significantly higher beam energies, SLAC E140 (an experiment dedicated to studying
the nuclear dependence of $R=\sigma_L/\sigma_T$) had used beam energies similar to JLab E03-103, but for
those measurements had not applied Coulomb corrections.  Once Coulomb corrections were applied to the E140
results, the conclusion that the was no evidence for a nuclear dependence of $R$ was less strong, with
$\Delta R=R_A-R_D$ about 1.5 standard deviations from zero.\footnote{Raphael: There is a double negative, I am 
not sure what is meant, but it should be corrected.} Further, when a combined analysis
of available SLAC E139, E140, and JLab E03-103 data was performed for data at $x=0.5$ and $Q^2\approx5$~GeV$^2$,
a similar deviation of $\Delta R$ was found~\cite{Solvignon:2009it}.

This result, combined with hints of a difference in $R$ for protons and deuterium motivated a 
new experiment to make further measurements of the nuclear dependence of $R=\sigma_L/\sigma_T$
with better precision than E140 and for a wider range of $x$ and $Q^2$ than previously
measured~\cite{12gev_nucr}.  This experiment will provide precise measurements of $R=\sigma_L/\sigma_T$
for the nucleon for $0.1<x<0.6$ and $1<Q^2<5~\text{GeV}^2$ as well as determination of $\Delta R= R_A-R_d$ for the same
kinematics using a copper target. Additional data will also be taken with carbon and gold targets for a subset
of the kinematics.


\subsection{Hadronization and Final State Effects}

The formation of hadrons and the propagation of quarks in a nuclear medium is important for interpeting the final states of reactions in SIDIS and Drell-Yan processes as well as critical in interpreting gluon distributions.  An important open question are the relative sizes of the interactions between asymptotic quark propagation and interactions after hadronizatoin.  As final states can reveal important information regarding the kinematics (such as Bjorken $x$) and flavor in a given interaction, improving the quality studies and models will simultaneously improve the extraction of modification data.  Studies have taken place at a variety of facilities, e.g. Fermilab~\cite{PhysRevC.75.035206}, JLab~\cite{PhysRevLett.99.242502, ELFASSI2012326}, and HERMES~\cite{Airapetian2011} as well as a future program with CLAS12 at JLab to study this in a broad set of channels~\cite{quarkformprop}.

Determination of the gluon distributions are also contaminated by final state interactions. nPDFs have been including routinely particle production in $d+\mathrm{Au}$ collisions into the fits, an observable that is very sensitive to both the initial state and hadronization of the gluon. The fragmentation functions for pion production have been best determined from $e^+e^-$ data, though the fragmentation of gluons have a sizable theoretical uncertainty. Data from the LHC at $7$~GeV\footnote{Raphael: TeV?} can not be well described by a global fit unless a significant cut in $p_{T}$ is applied to the data, leaving out a relevant portion of the covered $d+\mathrm{Au}$ region.  With the present data, this constitutes a sizable source of uncertainty for the extraction of the nPDFs and conclusions from incorporating particle production data from hadron colliders into the fits must be drawn carefully.

\subsection{Free Nucleon Parton Distributions}

While the free proton parton distributions have been studied extensively and with great precision, there still remain important measurements to be done to constrain the two leading-flavor parton distributions, in particular in the ratio of $d/u$ limit as $x \rightarrow 1$.  As these represent the basis of comparison for any nuclear modification effect, it is critical to have high quality data available, especially as one considers doing flavor decompositions of nuclei.  There are several programs which intend to improve the fixed-target lepton scattering data, such as using the ratio of ${}^{3}$H and ${}^{3}$He cross section~\cite{mar}, tagged spectator with deuterium~\cite{bonus12}, and parity-violating deep inelastic scattering on the proton~\cite{solid_pvdis}.  In addition, recent analyses of $W$ and $Z$ production in $p\bar{p}$ collisions from the CDF and D\O\ collaborations~\cite{D0:2014kma,Abazov:2013dsa,Acosta:2005ud,Aaltonen:2009ta,Aaltonen:2010zza,Abazov:2007jy} have also provided new constraints at large $x$ and have been incorporated into a recent global PDF analysis~\cite{Accardi:2016qay}.

