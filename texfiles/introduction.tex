\section{Introduction}

The simplest picture of the partonic structure of a nucleus is one comprised of an incoherent sum of the free protons and neutrons. Deep inelastic scattering experiments strongly suggest that such a picture is at best incomplete, and therefore to gain a more detailed understanding of the quark-gluon structure of nuclei a broad program in experiment and theory must be developed.  This includes novel measurements of nuclear structure with high energy leptonic probes with inclusive, semi-inclusive and exclusive final states, Drell-Yan processes with different incident hadrons on a variety of nuclei, a rigorous development of theoretical frameworks and modeling, and careful constraint and understanding of systematic effects.

There are several key questions to be answered that are within the reach of the physics community and would broadly expand upon our present knowledge of the role of quarks and gluons in nuclei.  This includes the nature of the isovector nuclear forces and their impact on the various parton distribution functions (PDFs) of nuclei, nuclear spin-dependent PDFs, the relation between the momenta of the bound internal quarks and the hadronic constituents, and the full femtoscopic imaging of the nucleus.  Coupled with these studies is the need for a rigorous formalism and a better understanding of the systematic effects in processes such as hadronization and in the extraction of nuclear longitudinal structure functions. 

The goal of the manuscript is to provide a status of the current experimental and theoretical understanding of the role of QCD in nuclei and to provide a road map containing a key set of questions for the next era of measurements and calculations. These new directions in experiment and theory will cover needed information for the latest nuclear parton distribution functions, programs which will study the spin and isospin-dependence of medium modification, better constrain both valence and sea distributions, and ultimately achieve a more complete tomography of the structure of nuclei. 

The structure of the manuscript is as follows: Sec.~\ref{sec:status} provides a summary of the current status of the EMC effect; Sec.~\ref{sec:nPDFs} discusses nuclear PDFs; Sec.~\ref{sec:directions} outlines some future directions for unpolarized lepton scattering measurements including measurements of short-ranged correlations; Secs.~\ref{sec:ivemc} and \ref{sec:pemc} discuss the isovector and polarized EMC effects; Sec.~\ref{sec:DY} discusses Drell-Yan measurements; Sec.~\ref{sec:light} looks at opportunities with light nuclei such as the deuteron; Sec.~\ref{sec:lf} discusses light-front methods; Sec.~\ref{sec:tagged} studies tagged reactions; Sec.~\ref{sec:GPDs} discusses nuclear GPDs; systematic effects are studied in Sec.~\ref{sec:systematics}, and finally a summary and outlook is given in Sec.~\ref{sec:conclusion}.
