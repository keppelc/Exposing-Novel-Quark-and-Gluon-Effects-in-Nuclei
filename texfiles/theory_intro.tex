\section{Opportunities with low $\mathbf{A}$ nuclei\label{sec:light}}
%
The difficulties in describing nuclei within QCD are myriad, e.g. all the challenges of understanding nucleon structure from QCD are inherited for nuclei, and usually amplified, and effective field theory approaches must deal with few- and many-body systems with numerous new energy scales. While QCD inspired models for nuclear structure can be applied across the table of nuclides~\cite{Cloet:2006bq,Cloet:2015tha,Stone:2017oqt,Stone:2016qmi}, lattice QCD 
%The primary difficulties of modeling nuclei within the QCD framework are the strong coupling constant which does not lend itself to perturbative calculations and the further complication of a many body system. Presently, even the most advanced 
techniques are currently only able to approach descriptions of nuclei with $A \leqslant 4$~\cite{Chang:2015qxa}, which is at the boundary of where one may expect large nuclear modification effects to occur. To move toward a more complete QCD description of nuclei advancement in QCD inspired models and {\it ab initio} techniques is critical, together with a robust comparison to few-body nuclear data. A key example going forward is provided by the formalism of generalized parton distributions, which can be extended to nuclei, and offer a new window of exploration which unifies the study of the medium modification of PDFs and form factors, and provides the opportunity to obtain a quark-gluon tomography of nuclei. 