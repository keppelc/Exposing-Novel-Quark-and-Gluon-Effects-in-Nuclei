\subsection{Poincar\'e Covariant Light-Front Spectral Function and Nuclear Structure}
{The Poincar\'e covariant spin-dependent spectral function, proposed in \cite{PhysRevC.95.014001, Pace:2013bq,Scopetta:2014yoa, Pace:2016eiq} and based on  the light-front (LF) Hamiltonian dynamics \cite{Dirac:1949cp, Keister:1991sb},  is a useful tool for a correct relativistic treatment  of nuclear structure, suitable for the study of deep inelastic scattering (DIS) or semi-inclusive deep inelastic scattering (SIDIS) processes at high momentum transfer \cite{mar,sid1,sid2}.}
 {Indeed the Bakamjian-Thomas construction \cite{Bakamjian:1953kh} of the Poincar\'e generators 
  allows one to embed the successful phenomenology for few-nucleon systems in a
Poincar\'e covariant framework.}

The LF spectral function for a three-fermion system  (as the $^3\mathrm{He}$ or a nucleon in valence approximation) is built up from the overlaps of the ground eigenstate of a proper mass operator \cite{PhysRevC.95.014001, Pace:2013bq,Scopetta:2014yoa, Pace:2016eiq} and 
the tensor product of a plane wave.
  %As shown by {{Keister and Polyzou}} such a state {{fulfills the macrocausality}}
{{The use of the light front nucleon momentum 
% in the intrinsic reference frame of the cluster (1,23) and the use  
%for the calculation of the LF spectral function 
%of  the tensor product of a plane  wave of momentum
%$\bm \kappa$ times the state of 
   %a fully interacting spectator subsystem 
   allows one  to take care of macrocausality \cite{Keister:1991sb} and  to introduce 
   {{a new effect of binding in the spectral function.}}}}
  % \\
% \vspace{1mm}
%\item {
%\Red 
%We have evaluated the Nucleon Spectral function
% for $^3He$, by approximating the IF overlaps with their non
% relativistic counterpart calculated  with the Av18 NN interaction}

The LF
% spin-dependent 
 spectral function  
%for a spin $1/2$ system composed by three fermions 
fulfills {normalization and momentum sum rule  at the same time. Integration of the spectral function on the energy 
$\epsilon$ 
of the pair yields the LF spin-dependent momentum distribution that
can  be expressed through 7 scalar functions  straightforwardly obtained from the system LF wave function as integrals on the relative momentum, of the spectator.
%\\
%An analogous expression occurs for the spin-dependent momentum distribution in terms of seven %functions ~~{{{${b}_{i,{\cal M}}\left [ |{\bf k}_{\perp}|,x, ({\bf  S}\cdot \hat {\bf k}_{\perp})^2,({\bf  S} \cdot %{\hat z})^2 \right] $}}}.
%\item
%\vspace{1mm}

{{Calculations of DIS 
%or SIDIS
 processes  based on a spectral function could indicate which is the gap with respect to the experimental data to be filled by effects of non-nucleonic degrees of freedom or by modifications of nucleon structure in nuclei.}}
%\subsubsection{EMC effect in {$^3$He} }
% {A first test of our approach is the EMC effect for $^3\mathrm{He}$.} 

% In Fig.~\ref{fig:pace} the ratio $R_2^{^3\mathrm{He}}/R_2^D$ is shown, with $R_2^A = A~ F_2^A/[Z ~F_2^p + (A-Z)~F_2^n]$.
%The  spectral function has been obtained from the non-relativistic wave function  with the Av18 NN interaction of \cite{Kievsky:1997bg}. As a first step, only
%the full expression for 
%the 2-body contribution 
%to the spectral function with the spectator pair  in a {{deuteron state}}
%has been used.
% \begin{figure}
%\centerline{\includegraphics[width=7.2cm]{plots/emc_018.eps}}
% \caption{Solid line: calculation with the {{LF Spectral Function}}. 
%\label{fig:pace}
%Dotted line: {convolution formula  with a momentum distribution as in \cite{Oelfke:1990uy}}. Only  the two-body contribution is considered.}
% \end{figure}

%{{Improvements clearly appear with respect to the
%convolution result with a momentum distribution as in \cite{Oelfke:1990uy}}.
%The next step will be the full calculation of the EMC effect for $^3\mathrm{He}$, including the exact
%3-body contribution. 
%}}

