\section{Summary and Key Issues\label{sec:conclusion}}
%
The area of medium modification in nuclei has been a rich field for several decades and here we have provided a survey of the status and the outstanding issues to be addressed.  To continue to advance this field in a coherent manner, a world-wide effort is required on a broad number of topics.  We identify some of the most urgent questions along with their programs.

\begin{itemize}
    \item{What is the isovector nature of the EMC effect?  This requires a set of high-precision experiments across nuclei of traditional inclusive cross section ratios, electroweak measurements which are sensitive to unique quark flavor combinations, and Drell-Yan experiments.  These would be complemented by exclusive isotope tagging experiments.}
    \item{What is the spin dependence of the EMC effect?  There is no experimental information available and any such measurement would break into new ground and would provide new information on potential mechanisms.}
    \item{What is the momentum-dependence and virtuality-dependence of the EMC effect?  This can start to be addressed through tagged scattering measurements, for both low and high nucleon momenta, which can separate mean field and local density mechanisms, respectively.}
    \item{What is the image of the full nucleus in terms of both quarks and gluons?  The study of generalized parton distributions through deeply virtual Compton scattering and deeply virtual meson production as well as at existing and future colliders are critical in producing a unified femtoscopic map of the nucleus.}
\end{itemize}

\begin{acknowledgments}
The authors are grateful to the ECT* in Trento, Italy for the support and for hosting the workshop ``Exposing Novel Quark and Gluon Effects in Nuclei'' in 2018 which made this work possible.  Argonne National Laboratory's work was supported by the U.S. Department of Energy, Office of Science, Office of Nuclear Physics, under contract DE-AC02-06CH11357.  
\end{acknowledgments}
