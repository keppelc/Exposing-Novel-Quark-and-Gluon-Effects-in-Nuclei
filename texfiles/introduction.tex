\section{Introduction}


Understanding the emergence of nuclei within QCD is a key challenge for modern science.  Since the advent of QCD, the role of quarks and gluons in nuclei relative to their counterparts in free protons and neutrons is unclear.  Traditional nuclear physics approaches the problem as comprised of many body hadronic states and has had significant success in describing the properties of nuclei across the chart of nuclides. However, because QCD is the fundamental theory of the strong interaction, it is unlikely that these hadron-level approaches can remain valid, or contain the correct degrees of freedom, for all processes at all energy scales. Clearly identifying these scales and processes is key to exposing the role of quarks and gluons in nuclei and thereby developing an understanding of how nuclei emerge within QCD.

The simplest picture of the partonic structure of the nucleus is one comprised of an incoherent sum of the free protons and neutrons.  This has been experimentally disproven and in order to gain a more detailed picture of the partonic structure of nuclei, a broad program must be developed on several fronts.  This includes novel measurements of nuclear structure with high energy leptonic probes, with inclusive, semi-inclusive and exclusive final states, Drell-Yan processes with different incident hadrons, a rigorous development of theoretical frameworks and modeling, and  careful constraint and understanding of systematic effects.

There are several key questions to be answered that are within the reach of the physics community and would broadly expand upon our present knowledge of medium modification.  This includes the nature of the isovector aspects of modification across many nuclei, the dependence on nuclear spin, the relation to the momenta of the bound internal quarks and hadronic constituents, and the full femtoscopic imaging of the nucleus.  Coupled with these studies is the need for a rigorous formalism and a better understanding of the systematic effects such as in hadronization and the nuclear longitudinal structure functions. 

In this paper, we summarize the current experimental and theoretical state of knowledge and put forth a road map and key set of questions for the next era of measurements and calculations.  These new directions in experiment and theory will cover needed information for the latest nuclear parton distribution functions, programs which will study the spin and isospin-dependence of modification, better constrain both valence and sea distributions, and ultimately achieve a more complete tomography of the structure of nuclei.

%\textbf{Notes from board}\\
%Introduction -- broad questions, why nuclei? \\
%Status  - Present, PDFs, SRC, Theory (basically Dave's)\\
%New Directions - IVEMC, SPIN, DY, Nuclear GPDs and light nuclei, Tagged\\
%Systematics - Hadroniztion, analysis of PDF, d/u\\
%Summary/Road Map - new key measurements, calculations, formalism\\
