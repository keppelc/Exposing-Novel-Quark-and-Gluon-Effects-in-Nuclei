\section{Poincar\'e Covariant Light-Front Spectral Function and Nuclear Structure\label{sec:lf}}
%
{The Poincar\'e covariant spin-dependent spectral function, proposed in \cite{PhysRevC.95.014001,Pace:2013bq,Scopetta:2014yoa,Pace:2016eiq} and based on  the light-front (LF) Hamiltonian dynamics \cite{Dirac:1949cp,Keister:1991sb},  is a useful tool for a correct relativistic treatment  of nuclear structure, suitable for the study of deep inelastic scattering (DIS) or semi-inclusive deep inelastic scattering (SIDIS) processes at high momentum transfer \cite{mar,sid1,sid2}.}
 {Indeed the Bakamjian-Thomas construction \cite{Bakamjian:1953kh} of the Poincar\'e generators
  allows one to embed the successful phenomenology for few-nucleon systems in a
Poincar\'e covariant framework.}

The LF spectral function
for a three-fermion system (as the $^3$He or a nucleon in valence approximation) depends on the  energy $\epsilon$ of the spectator subsystem  and on the LF  momentum $ \tilde{\bm \kappa}$ of the knocked out particle in the {{intrinsic reference frame of the (particle - spectator pair) cluster}}. It is built up from the overlaps of the ground eigenstate of a proper mass operator for the system \cite{PhysRevC.95.014001,Pace:2013bq,Scopetta:2014yoa,Pace:2016eiq}  and
  the tensor product of a plane wave for the particle
 times the fully interacting  state for the spectator.
 {{The use of the momentum {{${\tilde{\bm \kappa}}$}}
   allows one  to take care of macrocausality \cite{Keister:1991sb} and  to introduce
   {{a new effect of binding in the spectral function.}}}}

The LF
 spectral function
fulfills {normalization and momentum sum rule  at the same time. Integration of the spectral function on the energy
$\epsilon$
of the pair yields the LF spin-dependent momentum distribution that
can  be expressed through six scalar functions, straightforwardly obtained from the system LF wave function as integrals on the relative momentum of the spectator pair.


{{Calculations of DIS
 processes  based on a LF spectral function could indicate which is the gap with respect to the experimental data to be filled by effects of non-nucleonic degrees of freedom or by modifications of nucleon structure in nuclei.}}

