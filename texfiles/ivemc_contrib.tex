\section{Isovector EMC Effects\label{sec:ivemc}}
%
One aspect of the EMC effect that has not been fully explored are isovector-dependent effects.  Such effects occur in nuclei with $N \neq Z$, e.g. many of those shown in Fig.~\ref{fig:np_ratios}, and would necessarily include degrees of freedom beyond the density, $\rho$, or nuclear mass number $A$, which has long been a method of parametrization~\cite{Malace:2014uea}.  In this situation, the nuclear $u_A$ and $d_A$ distributions can be modified separately, for example in an isovector mean field background or due to a preference in nucleon flavors in short range correlation pairing.  In a flavor separation which makes the assumption that $u_p = d_n$ and $d_p = u_n$ in the modified system, this manifests itself as an apparent charge symmetry breaking, though only by the assumption that protons and neutrons retain their local charge symmetry identity. This assumption is not necessarily true for a bound system and can be invalid for asymmetric nuclei in the same vein as the binding energy has a symmetry energy component.  Continuing the binding energy analogy, as the symmetry energy is subleading to other bulk effects, such an isovector effect would be sub-leading to isoscalar modification effects.

The present world data have poor constraints on such an effect, in particular because many measurements of the EMC effect use symmetric or weakly asymmetric nuclei. One calculation~\cite{Cloet:2009qs,Cloet:2012td} using the Nambu-Jona-Lasinio (NJL) model and including the nuclear symmetry energy as an input, predicts deviations from the isoscalar EMC effect in the parton distribution functions on the order of several percent at large $x$.

Calculations based on SRCs predict a similar picture~\cite{Sargsian:2012sm, Arrington:2015wja}.  The observed correspondence of the short range correlation plateau with the slope of the EMC effect would indicate local densities as a driving mechanism~\cite{Weinstein:2010rt}.  Using the observation that proton-neutron pairs are found much more frequently than neutron-neutron or proton-proton pairs~\cite{Subedi:2008zz} and simple counting arguments, modification of protons and neutrons will be different in nuclei depending on the asymmetry.  In either a mean-field or SRC model, observation in the difference of quark flavors would require very high precision electromagnetic deep inelastic scattering measurements due to the suppression of the $d$ quark components weighted by the square of the electric charges. 

Weak interactions offer a novel method to probe flavor-dependent effects.  In charged-current processes, $u$ and $d$ quarks only participate in $W^-$ or $W^+$ exchange respectively. The NuTeV experiment~\cite{Zeller:2001hh} was carried out using neutrino beams at Fermilab and analyzed using the Paschos-Wolfenstein relation~\cite{Paschos:1972kj} to measure the weak mixing angle, $\sin^2\theta_\mathrm{W}$.  Due to the small neutrino cross section, heavy targets (iron) were employed which have a small neutron excess.  Charge symmetry in the bound nucleons was assumed and a significant deviation in $\sin^2\theta_\mathrm{W}$ was observed.  With the inclusion of an effect predicted by the NJL calculation noted above, much of the deviation is resolved~\cite{Cloet:2009qs,Bentz:2009yy}.  Additionally, possible tension has been noted in nuclear PDF fits between neutrino data and other data sets which probe different flavor combinations~\cite{Schienbein:2009kk}.

\subsection{Measuring Flavor Dependence with Parity-Violating DIS}
%
The interference between electromagnetic and neutral currents through parity-violating DIS provides a complementary process to pure electromagnetic DIS, and when used together provide a powerful method to access the flavor dependence of the EMC effect~\cite{Cloet:2012td}. In parity-violating DIS, a polarized lepton beam scattered from an unpolarized asymmetric nuclear target will form a small parity-violating cross-section between the two beam helicity states, $\sigma_{L,R}$, which at leading order is 
%
\begin{equation}
\frac{ \sigma_R - \sigma_L }{\sigma_R + \sigma_L} = -\frac{G_F\,Q^2}{4 \sqrt{2}\, \pi\, \alpha} 
\left[ Y_1(y)\,a_1(x) + Y_3(y)\,a_3(x) \right],
\label{eq:phy:apv}
\end{equation}
%
where $G_F$ is the Fermi constant, $\alpha$ the electromagnetic fine structure constant, $x = Q^2/(2M\nu)$ is the standard Bjorken-$x$ scaling variable, $M$ the mass the of nucleon, and $\nu$ the lepton energy transfer.  The $Y$ function is
%
\begin{equation}
Y_1(y) \approx 1, \qquad Y_3(y) \approx \frac{1 - (1-y)^2}{1 + (1-y)^2},
\end{equation}
%
and
%
\begin{align}
a_1(x) &= \frac{2\,\sum_q C_{1q}\, e_q\left[q(x) + \bar{q}(x)\right]}
{\sum_q\, e_q^2\left[q(x) + \bar{q}(x)\right]}, \\
a_3(x) &= \frac{2\,\sum_q C_{2q}\, e_q\left[q(x) - \bar{q}(x)\right]}
{\sum_q\, e_q^2\left[q(x) + \bar{q}(x)\right]},
\end{align}
%
where $y=\nu/E$, $E$ is the beam energy, $e_q$ is the quark electric charge couplings for flavor $q$, and $C_{1q}$ and $C_{2q}$ are the effective quark couplings dependent on the weak-mixing angle $\sin^2\theta_\mathrm{W}$~\cite{Patrignani:2016xqp}, with $C_{1u} \approx -0.19$ and $C_{1d} \approx 0.34$. The $a_1(x)$ term is dominant for fixed target, forward angle kinematics. 

The first predictions for $a_1(x)$ for $N \neq Z$ nuclei was made in Ref.~\cite{Cloet:2012td}. These calculations, which included a self-consistent isovector mean-field whose strength was fixed by the symmetry energy, found that $a_1(x)$ is particularly sensitive to flavor dependent nuclear effects such as a flavor dependent modification of the nuclear parton distribution functions. The solid line in Fig.~\ref{fig:ivemc:pvdis} presents the result $a_1(x)$ from Ref.~\cite{Cloet:2012td} for ${}^{48}$Ca, and the dashed curve is the result with no flavor dependent nuclear effects. An experiment has been proposed~\cite{emcpvdis} for the SoLID spectrometer at Jefferson Lab would be able to test the prediction from Ref.~\cite{Cloet:2012td} to better than 5-$\sigma$ with a ${}^{48}$Ca target. The projected errors for this experiment are shown in Fig.~\ref{fig:ivemc:pvdis}.

%===============================================================================
\begin{figure}[tbp]
\centering\includegraphics[width=\columnwidth]{plots/a1proj_2016.pdf}
\caption{Projected sensitivities of the quantity $a_1$ for a proposed parity-violating DIS experiment on a ${}^{48}$Ca target~\cite{emcpvdis}. The solid line is the full result from Ref.~\cite{Cloet:2012td} for $a_1$ in ${}^{48}$Ca, and the dashed line is the result when flavor-dependent nuclear effects are neglected.}
\label{fig:ivemc:pvdis}
\end{figure}
%===============================================================================





