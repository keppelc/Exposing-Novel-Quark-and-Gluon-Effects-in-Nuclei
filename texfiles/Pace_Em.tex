\subsection{Poincar\'e Covariant Light-Front Spectral Function and Nuclear Structure}
{The Poincar\'e covariant spin-dependent spectral function, proposed in \cite{PhysRevC.95.014001, Pace:2013bq,Scopetta:2014yoa, Pace:2016eiq} and based on  the light-front (LF) Hamiltonian dynamics \cite{Dirac:1949cp, Keister:1991sb},  is a useful tool for a correct relativistic treatment  of nuclear structure, suitable for the study of deep inelastic scattering (DIS) or semi-inclusive deep inelastic scattering (SIDIS) processes at high momentum transfer \cite{mar,sid1,sid2}.}
 {Indeed the Bakamjian-Thomas construction \cite{Bakamjian:1953kh} of the Poincar\'e generators 
  allows one to embed the successful phenomenology for few-nucleon systems in a
Poincar\'e covariant framework.}

The LF spectral function, ${{\cal P}^{\tau}_{{\cal M},\sigma'\sigma}({ \tilde {\bm \kappa}},\epsilon,S)}$,
for a three-fermion system with polarization $S$ (as the $^3\mathrm{He}$ or a nucleon in valence approximation) is built up from the overlaps $_{LF}\langle  \tau T ; 
\alpha,\epsilon ;J J_{z}; \tau\sigma,\tilde{\bm \kappa}|\Psi_{0}; S T_z \rangle$, where $|\Psi_{0}; S T_z \rangle$ is the ground eigenstate of a proper mass operator \cite{PhysRevC.95.014001, Pace:2013bq,Scopetta:2014yoa, Pace:2016eiq} and 
$|\tilde{\bm \kappa}, \sigma \tau; J_{} J_{z}; \epsilon,\alpha;T \tau \rangle_{LF}$ is the tensor product of a plane wave for particle 1 with LF momentum $~ \tilde{\bm \kappa} \equiv (\tilde{\kappa}^+,{\tilde{\bm \kappa}_\perp})$ in the {{intrinsic reference frame of the $[1+(23)]$ cluster}} 
 times the fully interacting  state for {the} $(23)$ pair with  energy eigenvalue $\epsilon$.
  %As shown by {{Keister and Polyzou}} such a state {{fulfills the macrocausality}}
{{The use of the nucleon momentum {{${\tilde{\bm \kappa}}$}}
% in the intrinsic reference frame of the cluster (1,23) and the use  
%for the calculation of the LF spectral function 
%of  the tensor product of a plane  wave of momentum
%$\bm \kappa$ times the state of 
   %a fully interacting spectator subsystem 
   allows one  to take care of macrocausality \cite{Keister:1991sb} and  to introduce 
   {{a new effect of binding in the spectral function.}}}}
  % \\
% \vspace{1mm}
%\item {
%\Red 
%We have evaluated the Nucleon Spectral function
% for $^3He$, by approximating the IF overlaps with their non
% relativistic counterpart calculated  with the Av18 NN interaction}

{ As shown in \cite{PhysRevC.95.014001, Pace:2013bq,Scopetta:2014yoa, Pace:2016eiq}, the LF
% spin-dependent 
 spectral function  
%for a spin $1/2$ system composed by three fermions 
fulfills {normalization and momentum sum rule  at the same time. Integration of the spectral function on the energy 
$\epsilon$ 
of the pair yields the LF spin-dependent momentum distribution that
can  be expressed through 7 scalar functions ~~ {{${b}_{i,{\cal M}}\left [ |{\bf k}_{\perp}|,x,({\bf  S}\cdot \hat {\bf k}_{\perp})^2,({\bf  S} \cdot {\hat z})^2 \right] $}}} ($i=0,6$), straightforwardly obtained from the system LF wave function as integrals on the relative momentum, $k_{23}$, of the pair.
%\\
%An analogous expression occurs for the spin-dependent momentum distribution in terms of seven %functions ~~{{{${b}_{i,{\cal M}}\left [ |{\bf k}_{\perp}|,x, ({\bf  S}\cdot \hat {\bf k}_{\perp})^2,({\bf  S} \cdot %{\hat z})^2 \right] $}}}.
%\item
%\vspace{1mm}

{{Calculations of DIS 
%or SIDIS
 processes  based on our spectral function could indicate which is the gap with respect to the experimental data to be filled by effects of non-nucleonic degrees of freedom or by modifications of nucleon structure in nuclei.}}
%\subsubsection{EMC effect in {$^3$He} }
 {A first test of our approach is the EMC effect for $^3\mathrm{He}$.} 

 In Fig.~\ref{fig:pace} the ratio $R_2^{^3\mathrm{He}}/R_2^D$ is shown, with $R_2^A = A~ F_2^A/[Z ~F_2^p + (A-Z)~F_2^n]$.
The  spectral function has been obtained from the non-relativistic wave function  with the Av18 NN interaction of \cite{Kievsky:1997bg}. As a first step, only
%the full expression for 
the 2-body contribution 
to the spectral function with the spectator pair  in a {{deuteron state}}
has been used.
%, while the 3-body contribution has been evaluated with an average $|{\bf k}_{23}|^2$.
 %Encouraging improvements clearly  appear with respect to a  convolution approach.
 %after comparing with experimental data.
%{ { { {Next step :  full calculation of the 3-body contribution}} }}
%The contribution from the {{2B channel}} 
%\vspace{2mm} 
 \begin{figure}
\centerline{\includegraphics[width=7.2cm]{plots/emc_018.eps}}
 \caption{Solid line: calculation with the {{LF Spectral Function}}. 
\label{fig:pace}
%Dashed line: as the solid one, but  with  $  \sqrt{\bar k^2_{23} } = ~ 136.37 ~ MeV$ for the deuteron. 
 %(AV18)
Dotted line: {convolution formula  with a momentum distribution as in \cite{Oelfke:1990uy}}. Only  the two-body contribution is considered.}
 \end{figure}
%{\Green The three curves, have been divided by the probability to find a deuteron in
%$^3He$, $\sim 2/3$.}
{{Improvements clearly appear with respect to the
convolution result with a momentum distribution as in \cite{Oelfke:1990uy}}.
% as in \cite{Sauer}. 
The next step will be the full calculation of the EMC effect for $^3\mathrm{He}$, including the exact
3-body contribution. 
%\large
%!
}}
%\subsection{T-even transverse momentum distributions }

%It can be shown that  in valence approximation the {{correlator}} ${{\Phi^{\tau}_{\alpha,\beta}(p,P,S)}} $ (see, e.g., \cite{Barone}) is related to the spin-dependent LF spectral function through the equation
%\be
%{{\Phi^{\tau}_{\alpha,\beta}(p,P,S)}} ~
%=~{2\pi ~ (P^+)^2 \over (p^+ )^2 ~ 4m } ~{ E_S \over {\cal{M}}_0[1,(23)]}
%\nonu
%\times~\sum_{\sigma\sigma '}\left \{~u_{\alpha}({\tilde {\bf p}},\sigma')~
%{{{\cal P}^{\tau}_{{\cal M},\sigma'\sigma}({ \tilde {\bm \kappa}},\epsilon,S)}}~
%{ \bar u}_\beta({\tilde {\bf p}},\sigma)~ \right \} \quad , \quad 
%\label{corr}
%\ee
%where ${\cal{M}}_0[1,(23)]$ is the free mass of the $[1+(23)]$ cluster and $E_S$ is the total energy of the $(23)$ pair in the intrinsic frame of the cluster.
%From Eq. (\ref{corr}) one can obtain the T-even transverse momentum distributions (TMD) in terms of the  scalar functions ${b}_{i,{\cal M}}$.
%{We intend {{to evaluate}} the TMD  for  {{$^3He$}}, that could be compared with those extracted from future {{measurements}}  of appropriate {{spin asymmetries}} in   {{$^3He(e,e'p)$}} experiments at high momentum transfer. }

%Linear equalities between the  transverse parton distributions were proposed (see, e.g. \cite{tmd})
%\be
% \hspace{-.5cm} \Delta f(x, |{\bf p}_{\perp}|^2 )=  \Delta'_T   f(x, |{\bf p}_{\perp}|^2 ) ~ + ~ {|{\bf p}_{\perp}|^2 \over 2 M^2}~ h^{\perp}_{1T}(x, |{\bf p}_{\perp}|^2 ) 
% \nonu
% g_{1T}(x, |{\bf p}_{\perp}|^2 ) = - h^{\perp}_{1L}(x, |{\bf p}_{\perp}|^2 )  \quad \quad \quad \quad 
%\ee
%From the LF spectral function one finds that these equalities
%hold exactly in valence approximation whenever the contribution to the transverse momentum distributions from the L=2 orbital angular momentum of the one-body off-diagonal density matrix is absent.

%As far as the quadratic relation discussed in the above papers is concerned
%\be
%(g_{1T} )^2  ~ + ~ 2 ~\Delta'_T   f ~ ~ h^\perp_{1T} = 0   \quad ,
%\ee
% in our approach it does not hold, even if  the contribution from the angular momentum $L=2$ is absent, because of the presence of {{$\int d k_{23}$}} in the expressions of the transverse momentum distributions.





